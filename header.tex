\title{
  NAMeC - Team Description Paper \\
  Small Size League RoboCup 2025 \\
  Application of Qualification in Division B
}

\titlerunning{NAMeC - Team Description Paper}

\author{
    M. Bernet, E. Bouvier, A. Calugi, B.Chew, P. Félix, J. Gautier, C. Godinat,
    C.Labbé, J. Lindois, T.W. Menier, C.A. Vlamynck
}

\authorrunning{T.W. Menier et al.}

\institute{IUT - Université de Bordeaux, Gradignan, France \\
\email{wanchai.menier@gmail.com} (corresponding author)
}

\maketitle

\renewenvironment{abstract}
{\begin{quote}
\noindent \par{\bfseries \abstractname}}
{\medskip\noindent
\end{quote}
}

\begin{abstract}
The following paper presents the advancements of NAMeC, french robotics team 
which has participated four times in the SSL league at Montreal, Sydney, Bordeaux, and Eindhoven.
  
As the architecture of the robot is being renewed \cite{tdp2023}, this paper will go
in detail about the different firmware and software improvements
that have been made to improve the quality of control system on our robots.
\end{abstract}

\keywords{RoboCup, Small Size League, embedded systems, path planning, decision making}

\section{Introduction}

Since joining the RoboCup Small Size League in 2018, the team has participated in four major competitions: Montreal, Sydney, Bordeaux, and Eindhoven.
The last competition is a significant milestone, as we transitioned to a fully student-led organization.

With our sights set firmly on the podium this year, we have implemented several critical improvements to enhance our performance.
These include the development of \textbf{BigBro}, a decision-making algorithm to optimize strategy, and \textbf{Trinamic},
a hardware solution for efficient Field-Oriented Control (FOC). Additionally, we have introduced advanced obstacle avoidance techniques,
implemented \textbf{Payload ACK} with TDMA for reliable communication, and designed a new wheel-fixing system to improve stability.
Other upgrades include the use of enhanced shell materials for durability and performance, as well as an innovative team color-changing
system for improved visibility and adaptability.

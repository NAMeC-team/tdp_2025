\title{
  NAMeC - Team Description Paper \\
  Small Size League RoboCup 2025 \\
  Application of Qualification in Division B
}

\titlerunning{NAMeC - Team Description Paper}

\author{
    A. Calugi, B.Chew, P. Félix, J. Gautier, C. Godinat,
    C.Labbé, J. Lindois, T.W. Menier, C.A. Vlamynck
}

\authorrunning{T.W. Menier et al.}

\institute{IUT - Université de Bordeaux, Gradignan, France \\
\email{wanchai.menier@gmail.com} (corresponding author)
}

\maketitle

\renewenvironment{abstract}
{\begin{quote}
\noindent \par{\bfseries \abstractname}}
{\medskip\noindent
\end{quote}
}

\begin{abstract}
The following paper presents the advancements of NAMeC, french robotics team 
which has participated four times in the SSL league. We will explain
in detail about the software improvements in terms of robot control, such as the
new decision-making system and optimal pass computation. Some improvements in the
firmware will also be reviewed.
\end{abstract}

\keywords{RoboCup, Small Size League, embedded systems, path planning, decision making}

\section{Introduction}

Since joining the RoboCup Small Size League in 2018, the team has participated in four major competitions: Montreal, Sydney, Bordeaux, and Eindhoven.
The last competition marks a significant milestone, as we transitioned to a fully student-led organization.
As the architecture of the robot is being renewed \cite{tdp2023}, the team has kept the current electronics.
The software has been improved with the implementation of a decision-making system, named "BigBro", which has been
used during the last competition. A score-based best pass location algorithm is also presented, which we plan to use
to make use of passes in a match. Additionally, we have introduced a new obstacle avoidance method,
that proved effective at last RoboCup. This paper also presents how we use functionalities of the radio module
used for communication, the nRF24L01+, to receive feedbacks from the robots while avoiding packet collisions.
\subsection{Collision-free communication}
Upon arriving at the competition during RoboCup 2024, an important problem
arised when we started testing our strategies. The robots would perform erratic
movements that were different from the orders sent.

The problem was that some packets arrived corrupted, because the packet
sending was not correctly implemented. As specified by the nRF24L01+ module's
datasheet, each packet has an air-time depending on its size (specified in seconds).
Thus meaning it is mandatory to wait such an amount of time before sending the next
packet.

As mentioned in our previous TDP, the original goal to enable reception of feedbacks while not causing any 
pakcet collision was to use Time-Distributed Multiple Access, a technique that
attributes a slot time to each device. Team TIGERs Manheim have described such an use \cite{tigers_tdp_2020}
to be viable for collision-less communication.

Instead of assigning a given time slot, we use the capabilities of the nRF24L01+ module,
that raises an interrupt once a packet is correctly transmitted
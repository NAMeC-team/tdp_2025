\subsection{Collision-free communication}
Upon arriving at the competition during RoboCup 2024, we fixed our radio communication
which emitted packets that seemed corrupted to the robots. This year, the communication has been made more
robust.

As mentioned in our previous TDP, the original goal to enable reception of feedbacks while not causing any 
pakcet collision was to use Time Distributed Multiple Access (TDMA). Team TIGERs Manheim have described such an use \cite{tigers_tdp_2020}
to be viable for collision-less communication. The idea behind TDMA is assigning a fixed slot time
for communication, such that packets for robot 0 are always sent in this time window.
Currently, speed control of our robots is done by continuously sending a packet
that contains the target speed for the robot to attain. So instead of implementing
TDMA, we have chosen to simply send packets sequentially,
with the capabilities of the nRF24L01+ module.


The radio module raises an interrupt pin on correct packet emission. Since auto-acknowledgement is enabled, 
as mentioned in the previous section, once this acknowledgement has been received by our emitting station,
or if the acknowledgement times out, the interrupt pin is raised.
After receiving the interrupt, we load and send the packet for the next robot, and immediately send it.
Because each packet is sent sequentially, this effectively creates a unique slot time for each robot.
This approach is very close to TDMA and is possible thanks to the payload ACK feature of the nRF24L01+.
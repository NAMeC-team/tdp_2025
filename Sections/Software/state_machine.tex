\subsection{Refactor of game state detection}
During last competition, many fouls were committed by our robots due to a
misunderstanding of the rulebook. The error could not be fixed easily because
the current codebase was parsing the current game state using a combination of conditions and ``j'' statements.
Thus, we have rewritten the state machine detection from scratch to improve maintenability
and readability of the code. Since our AI codebase is in Rust, we first searched for
librairies to implement the game states of a match. But because it is required
to handle rollback in the state machine, we have decided to not use any library,
to keep the codebase simple, as we do not need all of the functionalities of a finite state machine.
In particular, the state machine of a match runs infinitely (thus no accepting state).

The Rust language provides a match statement, which allows for pattern matching in the code.
Using this mechanism, only the transition function is required to correctly parse the game states
of the match. The transition function performs pattern matching over four variables:
current game state, latest command sent by the referee, reference poistion for the ball and latest
game event. With unit testing, this has proven sufficient to handle game states correctly,
and is much readable to the human eye, as pattern matching looks closely to listing the transitions
of the state machine on paper.

Note that the transition function is not called at every iteration of our complete system.
It is only called whenever a new command from the referee has been sent, or if the current
game state dynamic, meaning it depends on time or on another condition (such as the ``FreeKick'' state,
that transitions to ``Running'' once a robot touches the ball).
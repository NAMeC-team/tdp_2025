\subsection{Optimal pass computing}

Passing behaviour is one of the critical aspects of the SSL, allowing to perform
more complex behaviour. Thus, we have implemented a score-based search algorithm
where, given a rectangular region on the field, returns the best location towards
which the ball should be shot to.

Consider a robot that currently has the ball at location $p_{start}$, and would like to pass it to one
of its allies. Instead of selecting the best ally to perform the pass to,
we search for an optimal target location instead, expecting the nearest ally robot
to receive the ball. To do so, a rectangular search zone
is defined on the field, and using a given step value, a grid of evenly spaced points is generated inside it.
Each point generated defines a possible pass location, where the ball should approximately land after being shot.
To select the best one, a score is computed for each point $p_{pass}$,
based on the pass trajectory vector, enemy robots positions and velocities. By denoting the set of 
enemy robot positions as $S_{enemy}$, and ally positions as $S_{ally}$ the three conditions are as follows:

\begin{enumerate}
    \item \textbf{Enemy distance to pass trajectory}
    With $\overrightarrow{v_p} = p_{pass} - p_{start}$ the pass trajectory vector,
    and $\overrightarrow{v_e} = p_e - p_{start}$ the vector from the start location to the enemy,
    we project $\overrightarrow{v_e}$ onto $\overrightarrow{v_p}$ to compute this distance.
    This is performed for each enemy robot, giving the score computed by equation \ref{eq:enn-block-traj},
    with $proj(\overrightarrow{v_a}, \overrightarrow{v_a})$ denoting the vector projection of $\overrightarrow{v_a}$ onto $\overrightarrow{v_b}$, and $distance()$ being the euclidean distance function.
    
    \begin{equation}
        \label{eq:enn-block-traj}
        \sum_{p_{e} \in S_{enemy}}{\frac{-1}{(1 + distance(p_{start}, proj(\overrightarrow{v_e}, \overrightarrow{v_p})))}}
    \end{equation}

    The total sum is a malus to the final score, the malus being higher when the distance is smaller.
    Note that, if the resulting projected vector of an enemy position on the pass trajectory,
    lands outside of this trajectory, this computation is not performed and its score malus is set to 0.

    \item \textbf{Distance of the closest ally to the pass location}
    Naturally, the closer the ally is to the expected landing location, the less time will be required
    to catch the ball and manipulate it further. Due to electromagnetic interferences coming
    from our kicker mechanism, the maximum kicking power is limited, thus outgoing ball speed
    is not taken into account when calculating this score.
    The computed score is given by \ref{eq:dist-closest-ally}.
    
    \begin{equation}
        \label{eq:dist-closest-ally}
        k_{dca} \times \min_{p_a \in S_{ally}}{distance(p_a, p_{pass})}
    \end{equation}
    
    where $k_{dca}$ is a real coefficient used to scale the output score computed.

    \item \textbf{Enemy time of collison to the pass trajectory}

\end{enumerate}

%todo: equations for each
Figure \ref{graph_optimal_pass_example} explicits the different conditions on an example situation.

%\begin{figure}
%   \label{graph_optimal_pass_example}
%   \includegraphics{graph_pass.jpg}
%\end{figure}

% a score telling us how probable we can perform the pass properly
% based on three conditions
% - closest dist enemy to its projection on pass trajectory (minus robot radius)
% - pass distance (from robot to target pass location)

% TODO: replace 3rd condition with time step based collision detection (check every dt)
% - speed vector towards trajectory : log10(||v||) + 1.

% - display example on a graph + heatmap of score passing
% - explain improvements
% - add additional score heuristic that can make this useful for
%   best spot to shoot from to score
\subsection{Obstacle avoidance: exploration with left and right recursive search}
Before the competition of 2024, the team has implemented a new obstacle avoidance algorithm which has
proven to be effective at low driving speeds and with moving targets. This section will go in detail about how the implementation works.

When the robot wants to move to a specific position, he first checks if any other obstacle isn't in his way (direct line between him and the target), if there isn't an open path, it initiates a pathfinding algorithm to explore alternative routes. This exploration phase involves adjusting the robot's direction by incrementally rotating the vector facing the target, to the left and to the right, until a free angle is identified. The exploration tree is a binary tree with a branch going to the left and the other to the right of the robot.

\begin{itemize}
    \item \textbf{Obstacle Detection and Rotation}
    Upon detecting an obstacle, the robot begins exploring by rotating incrementally in one direction (e.g., left) (see figure \ref{fig:path_finding} for a visual example). After each small rotation, it evaluates whether the new direction results in a clear path of a short length. The robot checks if any obstacles intersect the trajectory.
    If no obstacles block the path, the angle is marked as “free”, and the robot proceeds along this trajectory.
    Otherwise if an obstacle remains, the robot continues rotating further in the same direction (left or right) until a free path is found or the maximum angle limit is reached.

    \item \textbf{Recursive Exploration}

    When a free path is identified, the algorithm recursively evaluates the next step from the new position. It continues checking whether this position brings the robot closer to the target while considering any new obstacles in its way. The exploration process is repeated until the target is reached or the maximum iteration number is reached.

    \item \textbf{Trajectory Smoothing}

    Once a path is calculated, it is smoothed to eliminate unnecessary waypoints, retaining only significant points to optimize the trajectory. It is performed by iterating over the points constituting the path, from the target to the robot, and check for each points if the robot can reach it without hitting an obstacle. If it can, the point is removed from the path, otherwise it is kept and we continue to the next point considering the current one as a new reference.

    \item \textbf{Optimal Direction Selection}

    If both left and right explorations yield free paths, the algorithm selects the optimal path based on distance to the target. If no fully clear path is found, the robot moves along the trajectory that gets it closest to the target.

    \item \textbf{Continuous Recalculation}

    The algorithm runs continuously, recalculating the best trajectory in real-time. Because of this, we just need to set the robot target to be the first point of the path found by the algorithm.
\end{itemize}

\begin{figure}
    \centering
    \includegraphics[scale=0.3]{path_finding.png}
    \caption{Path finding example}
    \label{fig:path_finding}
\end{figure}

\subsubsection{Comparison with Informed RRT*}
This obstacle avoidance algorithm has proven to be sufficient in practice. To compare its performance, we have decided to implement
a modification of the Informed RRT* path finding algorithm described by team KIKSbot \cite{tdp_kiksbot_2023}. Since RRT* is a popular choice in the SSL communinity,
choosing this algorithm seems to form a good comparison basis.

The Informed RRT* search was parameterized to perform 800 iterations, but because of its stochastic nature, a path might not be found
during these steps. In a real-life context, the path finding is run continuously during a match, so the search has been run 20 times
in order for the algorithm to give us the best result and lowest-path cost (in terms of distance traveled).
Note that no smoothing was performed on the Informed RRT* result.

\begin{figure}[h]
    \centering
    \begin{subfigure}[c][][c]{0.4\linewidth}
        \includegraphics[scale=0.45]{rstar}
        \caption{Binary-tree based search}
        \label{fig:own-obs-avoid}
    \end{subfigure}
    \hfill
    \begin{subfigure}[c][][c]{0.4\linewidth}
        \includegraphics[scale=0.45]{informed_rrt}
        \caption{Informed RRT* square search}
        \label{fig:informed-rrt-star}
    \end{subfigure}
    \label{fig:avoidance}
    \caption{Comparison of our own obstacle avoidance algorithm against Informed RRT*. Both axis represent the x and y coordinates.
    Informed RRT*. Parameters: step distance $d_{step} = 0.15$, and distance to check whether target is attained: $d_{attained} = 0.25$.
    The start and target points are respectively in blue and green,
    and grey points in the Informed RRT* graph represent the resulting tree generated.}
\end{figure}

Observing the results of both search algorithms, depicted in figures \ref{fig:own-obs-avoid} and \ref{fig:informed-rrt-star},
our algorithm finds a path of lower cost compared to Informed RRT*. The main advantage of our algorithm is that it is deterministic
and the next point search is always either towards the target, or to avoid an obstacle.
On static obstacles, we have determined experimentally that our algorithm often struggles between one intermediate target and another,
which is one of the main downsides of the path finding implemented. This was not a behavior that occured often during the SSL matches,
and we had difficulty recreating it in a testing environment. In conclusion, while both algorithms have their positive points,
the presented algorithm removes randomness, while still requiring to smooth the past in some cases. Stuttering could still be fixed by
not running the algorithm at every iteration of our main software.